\documentclass[14pt]{article}
\usepackage{listings}
\usepackage{color}
\usepackage{graphicx}
\usepackage{setspace}

\definecolor{dkgreen}{rgb}{0,0.6,0}
\definecolor{gray}{rgb}{0.5,0.5,0.5}
\definecolor{mauve}{rgb}{0.58,0,0.82}

\lstset{frame=tb,
  language=C,
  aboveskip=3mm,
  belowskip=3mm,
  showstringspaces=false,
  columns=flexible,
  basicstyle={\small\ttfamily},
  numbers=none,
  numberstyle=\tiny\color{gray},
  keywordstyle=\color{blue},
  commentstyle=\color{dkgreen},
  stringstyle=\color{mauve},
  breaklines=true,
  breakatwhitespace=true
  tabsize=3
}







\begin{document}
\title{\huge Sortare Topologica}
\date{\today}
\maketitle
\begin{center}
\vspace{30 mm}

\title{\huge Student: Boariu Dan-Andrei}
\\\vspace{10 mm}
\title{\huge Calculatoare si Tehnologia informatiei (Limba Romana)}
\\\vspace{10 mm}
\title{\huge Group CR 1.1A}
\\\vspace{10 mm}
\title{\huge Anul I}
\date{}
\maketitle

\newpage
\section*{Introduction}
\vspace{20 mm}
Sortarea topologică pentru un graf orientat aciclic(Directed Acyclic Graph -> DAG) este o ordonare liniară a vârfurilor astfel încât pentru fiecare legatura de varfuri uv, varful u vine înainte de v în ordonare. Sortarea topologică pentru un grafic nu este posibilă dacă graficul nu este un DAG.

\newpage
\end{center}
\section*{Enuntul problemei}
Topological sort. Implement two algorithms to determine the topological sort in
directed graphs, e.g., Kosaraju & Tarjan.




\newpage
\section*{Pseudocode} 
\\
Aici se afla pseudocode-ul  sortarii topologice:
\begin{lstlisting}
 

\end{lstlisting}

\newpage
\section*{Design-ul Aplicatiei}
\vspace{10 mm}
Libraria contine header-ul functii.h care contine header-ul functiei de sortare:
\\--void sort(int a[100][100])

\\\vspace{3 mm}
\\Topological Sort
\\ Codul meu se bazeaza pe introducerea unui numar specific de noduri si alcatuirea unei matrici de adiacenta prin care se specifica numarul de arce care intra in fiecare nod, deoarece sortarea topologica se aplica numai la DAG-uri, adica grafuri orientate si aciclice. 
indeg[i] semnifica numarul de arce care intra intr-un nod, astfel daca un nod are indeg[i]=0, se va porni de la acel nod pentru a realiza sortarea.
\newpage
\section*{Codul Sursa}
\begin{lstlisting}

//  main.c

#include <stdio.h>
#include <stdlib.h>

int main(){
    int i,j,k,n,a[200][200],adag[200],link[200],adauga=0;

    printf("Introduceti numarul de noduri:\n");
    scanf("%d",&n);

    printf("Introduceti matricea de adiacenta:\n");
    for(i=0;i<n;i++){
        printf("Enter row %d\n",i+1);
        for(j=0;j<n;j++)
            scanf("%d",&a[i][j]);
    }

    for(i=0;i<n;i++){
        adag[i]=0;
        link[i]=0;
    }

    for(i=0;i<n;i++)
        for(j=0;j<n;j++)
            adag[i]=adag[i]+a[j][i];

    printf("\n Ordinea topologica este :");

    while(adauga<n){
        for(k=0;k<n;k++){
            if((adag[k]==0) && (link[k]==0)){
                printf("%d ",(k+1));
                link[k]=1;
            }

            for(i=0;i<n;i++){
                if(a[i][k]==1)
                    adag[k]--;
            }
        }

        link++;
    }

    return 0;
}



\end{lstlisting}

\newpage
\section*{Experimente si rezultate}
\begin{lstlisting}
Introduceti numarul de noduri: 6
Introduceti matricea de adiacenta:
Enter row: 1
0 0 0 0 0 0
Enter row: 2
0 0 0 0 0 0
Enter row: 3
0 0 0 1 0 0
Enter row: 4
0 1 0 0 0 0
Enter row: 5
1 1 0 0 0 0
Enter row: 6
1 0 1 0 0 0

The topological order is: 5,6,1,3,4,2


\end{lstlisting}


\newpage
\section*{Concluzii}
\vspace{20 mm}
Lucrand la acest proiect cu sortare topologica, am realizat ca nu stiam asa de multe pe cat credeam despre grafuri si m-a ajutat sa inteleg mai bine cum functioneaza anumiti algoritmi asupra grafurilor. Folosindu-ma de Topological Sort am reusit sa lucrez pe un graf cu un anumit numar de noduri si sa imi reamintesc implementarea matricei de adiacenta pentru aceste noduri.

\\\vspace{20mm}
\section*{Referinte}
\large 1)http://www.sharelatex.com

\end{document}
